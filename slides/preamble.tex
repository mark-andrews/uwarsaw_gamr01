\RequirePackage{pifont,manfnt}
\RequirePackage{booktabs}
\usepackage{subcaption}
\RequirePackage[T1]{fontenc}
\RequirePackage{mathpazo}
\RequirePackage{eulervm}
\linespread{1.05}
\RequirePackage{tikz}
\RequirePackage{tikz-cd}
\usepackage{pgfplots}
\usetikzlibrary{arrows,positioning,matrix} 
\RequirePackage{xspace}
\RequirePackage{apacite}
\RequirePackage{rotating}
\RequirePackage{multirow}
\usepackage{fontawesome}
\usepackage{nth}
\pgfplotsset{compat=1.16}
\newcommand{\Prob}[1]{\mathrm{P}( #1 )}
\newcommand{\dcat}[1]{\mathrm{dcat}( #1 )}
\newcommand{\ddirichlet}[1]{\mathrm{ddirichlet}( #1 )}
\newcommand*{\given}{\vert}
\newcommand{\hdpmm}{\textsc{hdptm}\xspace}
\newcommand{\bnc}{\textsc{bnc}\xspace}
\newcommand{\brms}{Brms\xspace}
\newcommand{\mcmc}{\textsc{mcmc}\xspace}
\newcommand{\icc}{\textsc{icc}\xspace}
\newcommand{\reml}{\textsc{reml}\xspace}
\newcommand{\mad}{\textsc{mad}\xspace}

\newcommand\iidsim{\mathrel{\overset{\makebox[0pt]{\mbox{\normalfont\tiny iid}}}{\sim}}}
\newcommand\defeq{\mathrel{\overset{\makebox[0pt]{\mbox{\normalfont\tiny def}}}{=}}}
\newcommand{\hpd}{\textsc{hpd}\xspace}
\newcommand{\Probc}[1]{\mathrm{P}_{\text{\!\tiny \textsc{c}}}( #1 )}
\newcommand{\Proba}[1]{\mathrm{P}_{\text{\!\tiny \textsc{a}}}( #1 )}
\newcommand{\wnew}{w_{j}}
\newcommand{\wjinew}{w_{ji}}
\newcommand{\pinew}{\pi_{j}}
\newcommand{\data}{\mathcal{D}}
\newcommand{\dic}{\textsc{dic}\xspace}
\newcommand{\studentt}[1]{t_{#1}}
\setbeamerfont{title}{family=\it}
\setbeamerfont{frametitle}{family=\it}

\RequirePackage{tikz}
\usetikzlibrary{trees}
\usetikzlibrary{matrix}

\RequirePackage{amssymb,latexsym,amsmath,amsfonts,amscd}

\usecolortheme[named=gray]{structure} 
\setbeamercolor{titlelike}{fg=black!60!red}
\definecolor{Mygrey}{gray}{0.75}

\newcommand{\rreallytiny}{\fontsize{3}{3}\selectfont}
\newcommand{\reallytiny}{\fontsize{5}{5}\selectfont}

\usetikzlibrary{decorations.pathmorphing} % noisy shapes
\usetikzlibrary{fit}					% fitting shapes to coordinates
\usetikzlibrary{backgrounds}	% drawing the background after the foreground
\usetikzlibrary{matrix}

\tikzstyle{background}=[rectangle, fill=none,
						draw=black,
                                                inner sep=0.3cm,
                                                rounded corners=3mm]

\tikzstyle{observation}=[circle,font=\small,minimum size=5mm,inner sep=0mm,
                                    draw=black!70,
                                    fill=black!10]

\tikzstyle{state}=[circle,font=\small,minimum size=5mm,inner sep=0mm,
                                   draw=black!70,
                                    fill=none]

\tikzstyle{limit}=[rectangle,font=\small,minimum size=0mm,inner sep=0mm,
                                    fill=none]

\tikzstyle{parameter}=[circle,font=\small,minimum size=5mm,inner sep=0mm,
                                   draw=black!70,
                                    fill=none]
% tikz stuff
\usepackage{tikz}
\usetikzlibrary{shapes}
\usetikzlibrary{positioning,shapes,trees,arrows,shadows,arrows.meta,backgrounds,fit}

\tikzset{every path/.style={-latex,thick}}

\tikzset{
  basic/.style = {font=\sffamily},
  % material/.style  = {basic, text width=8mm, font=\footnotesize\sffamily, fill=yellow!60},
  % revision/.style  = {material, fill=blue!30},
  % root/.style   = {basic,  align=center, fill=pink!60},
  level 1/.style = {basic, align=center, sibling distance = 30mm},
  level 2/.style = {basic, sibling distance = 20mm},
  level 3/.style = {basic, sibling distance = 15mm},
  % level 4/.style = {level distance=10mm,basic, fill=pink!60, sibling distance = 20mm}
}


\graphicspath{{../images/}}

\DeclareSymbolFont{legacymaths}{OT1}{cmr}{m}{n}
\DeclareMathAccent{\dot}     {\mathalpha}{legacymaths}{95}
\DeclareMathAccent{\bar}     {\mathalpha}{legacymaths}{22}
\DeclareMathAccent{\tilde}     {\mathalpha}{legacymaths}{126}

\setbeamertemplate{footline}[frame number]
